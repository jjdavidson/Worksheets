\documentclass[12pt]{article}

\usepackage{amsmath}
\usepackage{amsthm}
\usepackage{graphicx}

\newtheorem*{note}{Note}
\numberwithin{equation}{section}
\graphicspath{ {./images} }

\title{Polynomial Division Tutorial}
\author{Jonathan Davidson}
\date{}

\begin{document}
\maketitle

\section{Introduction}
Polynomials are used in almost every field of mathematics and have applications to every field of engineering and science. The ubiquity of polynomials in these fields stems from the rich algebraic structure that polynomials encapsulate. Similar to integers, polynomials can be added, subtracted, and multiplied. The result of these computations will be another polynomial. For example,
\begin{align*}
(x^3-2x^2-x-1) + (x^2-1) &= x^3-x^2-x-1 \\
(x^3-2x^2-x-1) - (x^2-1)&= x^3-3x^2-x \\
(x^3-2x^2-x-1) \times (x^2-1) &= x^5-2x^4-2x^3+x^2+x+1
\end{align*}
However, division introduces some complexity. \textbf{Division produces two objects, a quotient and a remainder.}
\section{Division Over Integers}
Consider the division $130 \div 7$. It has a quotient of 18 and a remainder of 4. This relationship can be restated with subtraction and multiplication as
\begin{equation}
130 - \boxed{18} \times 7 = 4
\end{equation}
where the quotient has been indicated by a box outline. It is also possible to write
\begin{equation}
130 - \boxed{10} \times 7 = 60
\end{equation}
It is tempting to say that the quotient is 10 and the remainder is 60. To prevent  ambiguity, \textbf{remainders are required to be less than the divisor}. However, Equation 2.2 can be useful in the computation of the quotient and remainder. Dividing the remainder again by the divisor 7
\begin{align*}
130 - \boxed{10} \times 7 &= 60 \\
60 - \boxed{8} \times 7 &= 4
\end{align*}
Adding up these equations yields equation 2.1. Observe that the sum of the  intermediate quotients 10 and 8 yield the true quotient 18. The number on the right side of the last equation is the remainder.

\vspace{1em}
\fbox{\begin{minipage}{0.9\textwidth}
\textbf{Note:} Using a calculator on the expression $130 \div 7$ yields the decimal $18.\overline{571428}$. There is an analogy to decimals within polynomials called power series, but the computations of either decimals or power series require understanding of the division operation in terms of quotients and remainders.
\end{minipage}}
\vspace{1em}

\noindent Consider the example $34921 \div 124$.
\begin{align*}
34921 - \boxed{200} \times 124 &= 10121 \\
10121 - \boxed{50} \times 124 &= 3921 \\
3921 - \boxed{30} \times 124 &= 201 \\
201 - \boxed{1} \times 124 &= 77
\end{align*}
The quotient is $200+50+30+1 = 281$ and the remainder is $77$. During the algorithm, \textbf{the intermediate remainder terms shrink until it is less than the divisor} 124. This guarantees that the process will eventually terminate. This method is more flexible than the long division algorithm taught in elementary school. This method even can be used to do long division in different number systems.

\section{Division Over Polynomials}
In order to apply the \textbf{division algorithm} from the previous section to polynomials, there needs to be a way to compare the sizes of polynomials. In the division algorithm, the intermediate remainders became smaller and smaller in the context of numbers. For polynomials, the \textbf{degree of the polynomial will determine the size of a polynomial}. Recall the degree $\deg[p(x)]$ of a polynomial $p(x)$ is the highest power of the terms.

Consider the polynomials $x^3-2x^2-x-1$ and $x^2-1$ from Section 1. The goal is to find polynomials $q(x)$ and $r(x)$ such that 
\begin{equation}
(x^3-2x^2-x-1) - \boxed{q(x)} (x^2-1) = r(x)
\end{equation}
such that the degree of $r(x)$ is less than the degree of $x^3-2x^2-x-1$. The easiest way to do this is to chose $q(x) = x$ in order to eliminate the $x^3$ term. Equation 3.1 becomes
\begin{equation}
(x^3-2x^2-x-1) - \boxed{x} (x^2-1) = -2x^2-1
\end{equation}
Applying the full division algorithm
\begin{align*}
(x^3-2x^2-x-1) - \boxed{x} (x^2-1) &= -2x^2-1 \\
(-2x^2-1) - \boxed{-2}(x^2-1) &= 1
\end{align*}
The quotient is the sum of the boxed numbers $x-2$ and the remainder is 1.

Let $p(x)$ be the dividend, $q(x)$ be the quotient, $d(x)$ be the divisor, and $r(x)$ be the remainder. Every step of the process will take the form
\begin{equation}
p(x) - q(x)d(x) = r(x)
\end{equation}
The \textbf{polynomial division algorithm} is summarized as 
\begin{enumerate}
\item Multiply the divisor $d(x)$ by a monomial $q(x) = ax^n$ in order to cancel out with the leading term of the dividend $p(x)$.
\item Subtract the product $q(x)d(x)$ from the dividend $p(x)$ to get the remainder $r(x)$.
\item If the intermediate remainder $\deg[r(x)] \geq \deg[d(x)]$ repeat the process with $r(x)$ as the new dividend and $d(x)$ as the divisor.
\item If the $\deg[r(x)] < \deg[d(x)]$, $r(x)$ is the remainder and the quotient is the sum of all the boxed monomials.
\end{enumerate}
Another example will illustrate the polynomial division algorithm in action.

\section{Polynomial Division Example}
Let $p(x) = x^6+3x^5-4x^4-8x^2+x+7$ and $d(x) = x^3-2x+1$. The division algorithm yields
\begin{align*}
(x^6+3x^5-4x^4+7) - \boxed{x^3} (x^3-2x+1) &= 3x^5-2x^4-x^3+7 \\
(3x^5-2x^4-x^3+7) - \boxed{3x^2} (x^3-2x+1) &= -2x^4+5x^3-3x^2+7 \\
(-2x^4+5x^3-3x^2+7) - \boxed{-2x} (x^3-2x+1) &= 5x^3-7x^2+2x+7 \\
(5x^3-7x^2+2x+7) - \boxed{5} (x^3-2x+1) &= -7x^2+12x+2
\end{align*}
The process stops since the remainder term has degree 2 which is less than the divisor which has degree 3. Summing the boxed terms yields the quotient $q(x) = -7x^2+12x+2$. The remainder is $r(x) = -7x^2+12x+2$. 

\section{Synthetic Division}
In the special case where the divisor is of the form $(x-c)$, the division algorithm can be written in a compact form. Consider the polynomial division $(2x^4-3x^3-4x^2+x+7) \div (x-2)$. The standard division algorithm yields
\begin{align*}
(2x^4-3x^3-4x^2+x+7) - \boxed{2x^3}(x-2) &= x^3-4x^2+x+7 \\
(x^3-4x^2+x+7) - \boxed{x^2}(x-2) &= -2x^2+x+7 \\
(-2x^2+x+7) - \boxed{-2x}(x-2) &= -3x+7 \\
(-3x+7) - \boxed{-3}(x-2) &= 1
\end{align*}
The quotient is $q(x) = x^2-2x-3$ and the remainder is $r(x) = 1$.

At each step of the process, the dividend polynomial only makes two changes: the leading term is removed, and the coefficient of the next leading term is modified. Synthetic division 

\end{document}
